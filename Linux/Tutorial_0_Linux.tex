% ==============================================
% CFD Tutorial Template: OpenFOAM Terminal Basics
% Class: scrartcl (KOMA-Script) - optimized for tutorials
% Content: pitzDaily with simpleFoam
% ==============================================

\documentclass[
    parskip=half,      % Space between paragraphs (better for steps)
    DIV=12,            % Optimal margins for readability
    headings=normal,   % Professional heading sizes
    fontsize=11pt,     % Slightly larger for screen reading
    english            % Language setting
]{scrartcl}

% ========== ESSENTIAL PACKAGES ==========
\usepackage[utf8]{inputenc}
\usepackage[T1]{fontenc}
\usepackage{lmodern}
\usepackage{geometry}
\usepackage{tikz}
\usepackage{microtype}           % Superior typography
\usepackage{graphicx}            % For screenshots (optional)
\usepackage{xcolor}              % Color coding
\usepackage{listings}            % Code listings
\usepackage{siunitx}             % Proper units: \si{\meter\per\second}
\usepackage{amsmath}             % Math equations
\usepackage{hyperref}            % Clickable links
\usepackage{url}
\usepackage{booktabs}            % Professional tables
\usepackage{caption}             % Better figure captions
\usepackage[english]{babel}
\usepackage{float}

% ========== CUSTOM COLORS ==========
\definecolor{codebg}{RGB}{240, 245, 250}
\definecolor{cmdcolor}{RGB}{0, 100, 0}
\definecolor{outputcolor}{RGB}{100, 100, 150}
\definecolor{warningcolor}{RGB}{180, 0, 0}
\definecolor{tipcolor}{RGB}{0, 100, 50}

% ========== CODE LISTING SETUP ==========
\lstset{
	backgroundcolor=\color{codebg},
	basicstyle=\ttfamily\small,
	breaklines=true,
	frame=single,
	captionpos=b,
	rulecolor=\color{gray!30},
	tabsize=2,
	showstringspaces=false
}

% Custom style for terminal commands
\lstdefinestyle{terminal}{
	basicstyle=\ttfamily\small\color{cmdcolor},
	backgroundcolor=\color{black!5},
	frame=none,
	columns=fullflexible
}

% Custom style for OpenFOAM files
\lstdefinestyle{openfoam}{
	language=C++,
	morekeywords={dimensions, internalField, boundaryField, type, value},
	keywordstyle=\color{blue},
	commentstyle=\color{gray!50}\itshape
}

\lstdefinestyle{output}{
	backgroundcolor=\color{black!5},
	basicstyle=\ttfamily\small\color{outputcolor},
	frame=none,
	breaklines=true
}
% ========== HYPERREF SETUP ==========
\hypersetup{
	colorlinks=true,
	linkcolor=blue,
	citecolor=green,
	filecolor=magenta,
	urlcolor=cyan,
	pdftitle={OpenFOAM Tutorial: pitzDaily with simpleFoam},
	pdfauthor={Your Name},
	pdfsubject={CFD with OpenFOAM},
	pdfkeywords={OpenFOAM, CFD, simpleFoam, pitzDaily, terminal}
}


\newcommand{\warning}[1]{\par\vspace{5pt}\noindent\textcolor{red}{\textbf{WARNING:}} #1\par\vspace{3pt}}
\newcommand{\tip}[1]{\par\vspace{5pt}\noindent\textcolor{green!50!black}{\textbf{TIP:}} #1\par\vspace{3pt}}
\newcommand{\important}[1]{\par\vspace{5pt}\noindent\textcolor{blue}{\textbf{IMPORTANT:}} #1\par\vspace{3pt}}                    

% ========== CUSTOM COMMANDS ==========
\newcommand{\terminalcmd}[1]{\textbf{\texttt{\textcolor{cmdcolor}{#1}}}}
\newcommand{\ofkeyword}[1]{\textbf{\textcolor{blue}{#1}}}
\newcommand{\filename}[1]{\texttt{\textcolor{purple}{#1}}}
\newcommand{\commandexample}[1]{\vspace{5pt}\hspace{10pt}\terminalicon\ \terminalcmd{#1}\vspace{5pt}}
\newcommand{\key}[1]{\texttt{\textbf{#1}}}


% ========== DOCUMENT START ==========
\begin{document}
	
	\title{Tutorial \#0: Linux Fundamentals for CFD}
	\subtitle{Getting Started with Ubuntu 24.04, Bash Shell, and Git}
	\subject{Open Source CFD}
	\author{Robert Castilla \\ Department of Fluid Mechanics}
	\date{2025-26 \\ Version 1.0}
	\publishers{Learning Objectives:
		\begin{itemize}
			\item Install and setup linux distribution Ubuntu
			\item Get familiar with bash shell and principal commands
			\item Write and run a simple bash script
			\item Setup and use git as a version control system
	\end{itemize}}
	
	\maketitle
	
	\begin{abstract}
		This tutorial introduces the essential Linux skills needed for the CFD course. You'll learn to set up Ubuntu 24.04, navigate the filesystem using the bash shell, write simple scripts, and use Git for version control. These foundations will prepare you for working with OpenFOAM, FreeCAD, and managing your CFD projects effectively.
	\end{abstract}
	
	
	\tableofcontents
	\newpage
	
	% ========== SECTION 1: UBUNTU 24.04 ==========
	\section{Getting Started with Ubuntu 24.04 LTS}
	
	\subsection{What is Ubuntu?}
	Ubuntu is a Linux distribution based on Debian, known for being user-friendly 
	andits  extensive community support. Version 24.04 is a \textbf{Long Term Support (LTS)} release, meaning it receives security updates and maintenance until 2029.
	
	\textbf{Why Ubuntu for CFD?}
	\begin{itemize}
		\item OpenFOAM is natively supported on Ubuntu
		\item FreeCAD and CfdOF work seamlessly
		\item Extensive documentation and community help
		\item Package manager (APT) simplifies software installation
	\end{itemize}
	
	\subsection{Installation Options}
	
	You have two main ways to get Ubuntu:
	
	\begin{enumerate}
			\renewcommand{\labelenumi}{\textbf{Step \arabic{enumi}:}}
		
		\item \textbf{Dual Boot (Recommended)}: Install alongside Windows
		\begin{itemize}
			\item Best performance for CFD simulations
			\item Access to all system resources
			\item Requires partitioning your hard drive
		\end{itemize}
		
		\item \textbf{Virtual Machine}: Run Ubuntu inside Windows
		\begin{itemize}
			\item Easier setup (using 
			\hyperlink{https://www.virtualbox.org/}{VirtualBox} or 
			\hyperlink{https://canonical.com/multipass}{Multipass})
			\item Slightly slower for heavy computations
			\item Good for testing and learning
		\end{itemize}
		
	\end{enumerate}
	
	\tip{You have already available an \filename{ova} file that can be used to create a
	VirtualBox virtual machine.}
	
	\subsection{Creating a Bootable USB Stick}
	
	\begin{enumerate}
		\item \textbf{Download Ubuntu}: Get the ISO from \url{https://ubuntu.com/download}
		\item \textbf{Create bootable USB}:
		\begin{itemize}
			\item \textbf{Windows}: Use \href{https://rufus.ie}{Rufus}
			\item \textbf{Linux}: Use \terminalcmd{dd} command or \texttt{Startup Disk Creator}
			\item \textbf{macOS}: Use \href{https://www.balena.io/etcher}{balenaEtcher} 
		\end{itemize}
	\end{enumerate}
	
	\tip{Rufus (Windows) or balenaEtcher (macOS) are the most user-friendly options.}
	
	\subsection{Installation Steps}
	
	\begin{enumerate}
		\item Boot from USB (press \key{F12}, \key{F2}, or \key{Esc} during startup)
		\item Select \textbf{"Try or Install Ubuntu"}
		\item Choose language and keyboard layout
		\item Connect to WiFi (optional but recommended)
		\item Select \textbf{"Extended selection"} for more pre-installed apps
		\item Enable \textbf{"Install third-party software"} for drivers and codecs
		\item Choose installation type:
		\begin{itemize}
			\item "Install alongside Windows" for dual boot
			\item "Erase disk and install Ubuntu" for clean install
		\end{itemize}
		\item Create user account and password
		\item Wait for installation to complete and reboot
	\end{enumerate}
	
	\warning{Back up your data before partitioning or installing any operating system!}
	
	\subsection{Post-Installation Setup}
	
	After first boot, run these essential updates:
	
	\begin{lstlisting}[style=terminal, caption=Initial system update]
		# Update package lists
		sudo apt update
		
		# Upgrade installed packages
		sudo apt upgrade
		
		# Update snap packages
		sudo snap refresh
		
		# Clean up unnecessary packages
		sudo apt autoremove
		sudo apt autoclean
	\end{lstlisting}
	
	\important{Always run \terminalcmd{sudo apt update} before installing new software to ensure you get the latest versions.}
	
	\subsection{Installing Essential Software}
	
	\begin{lstlisting}[style=terminal, caption=Install basic tools]
		# Build essentials (compilers, make, etc.)
		sudo apt install build-essential
		
		# Text editor
		sudo apt install vim nano
		
		# System monitoring
		sudo apt install htop btop neofetch
		
		# File utilities
		sudo apt install tree ncdu
		
		# Compression tools
		sudo apt install unzip p7zip-full
	\end{lstlisting}
	
	\tip{Use \terminalcmd{htop} to monitor system resources while running CFD simulations.}
	
	% ========== SECTION 2: BASH SHELL ==========
	\section{The Bash Shell}
	
	\subsection{What is a Shell?}
	The shell is a command-line interface that lets you interact with your operating system. \textbf{Bash} (Bourne Again SHell) is the default shell in Ubuntu and most Linux distributions.
	
	\subsection{Opening the Terminal}
	\begin{itemize}
		\item Press \key{Ctrl+Alt+T}
		\item Search for "Terminal" in applications menu
		\item Right-click desktop → "Open in Terminal"
	\end{itemize}
	
	\subsection{Navigating the Filesystem}
	
	\begin{table}[h]
		\centering
		\begin{tabular}{lll}
			\toprule
			\textbf{Command} & \textbf{Meaning} & \textbf{Example} \\
			\midrule
			\terminalcmd{pwd} & Print Working Directory & \terminalcmd{pwd} → \texttt{/home/username} \\
			\terminalcmd{ls} & List directory contents & \terminalcmd{ls -la} (detailed view) \\
			\terminalcmd{cd} & Change Directory & \terminalcmd{cd Documents} \\
			\terminalcmd{cd ..} & Go up one level & \terminalcmd{cd ../..} (up two levels) \\
			\terminalcmd{cd \textasciitilde} & Go to home directory & \terminalcmd{cd \textasciitilde} \\
			\terminalcmd{cd -} & Go to previous directory & \terminalcmd{cd -} \\
			\bottomrule
		\end{tabular}
		\caption{Essential navigation commands}
	\end{table}
	
	\begin{lstlisting}[style=terminal, caption=Practice navigation]
		# Where am I?
		pwd
		
		# What's in this directory?
		ls -la
		
		# Go to Documents
		cd ~/Documents
		
		# Create a new directory
		mkdir cfd_tutorials
		
		# Go into it
		cd cfd_tutorials
		
		# Go back home
		cd
	\end{lstlisting}
	
	\subsection{File Operations}
	
	\begin{table}[h]
		\centering
		\begin{tabular}{lll}
			\toprule
			\textbf{Command} & \textbf{Purpose} & \textbf{Example} \\
			\midrule
			\terminalcmd{cp} & Copy & \terminalcmd{cp <file1> <file2>} \\
			\terminalcmd{mv} & Move/Rename & \terminalcmd{mv <oldname> <newname>} \\
			\terminalcmd{rm} & Remove & \terminalcmd{rm <file>} \\
			\terminalcmd{rm -r} & Remove directory & \terminalcmd{rm -r <folder>} \\
			\terminalcmd{mkdir} & Create directory & \terminalcmd{mkdir <folder>} \\
			\terminalcmd{touch} & Create empty file & \terminalcmd{touch <file>} \\
			\terminalcmd{cat} & Display file & \terminalcmd{cat <file>} \\
			\terminalcmd{less} & View file page by page & \terminalcmd{less <file>} \\
			\bottomrule
		\end{tabular}
		\caption{File and directory operations}
	\end{table}
	
	\warning{Be careful with \terminalcmd{rm -r}! It permanently deletes files without a trash bin.}
	
	\subsection{Understanding Paths}
	
	\begin{itemize}
		\item \textbf{Absolute path}: Starts from root directory
		\begin{lstlisting}[style=terminal]
			/home/username/Documents/cfd_tutorials
		\end{lstlisting}
		
		\item \textbf{Relative path}: Relative to current location
		\begin{lstlisting}[style=terminal]
			Documents/cfd_tutorials  # If you're in /home/username
			../Downloads            # One level up, then into Downloads
		\end{lstlisting}
	\end{itemize}
	
	\tip{Use \key{Tab} for auto-completion! Type the first few letters and press \key{Tab}.}
	
	\subsection{Useful Shortcuts}
	
	\begin{table}[H]
		\centering
		\begin{tabular}{ll}
			\toprule
			\textbf{Shortcut} & \textbf{Function} \\
			\midrule
			\texttt{Ctrl+C} & Cancel current command \\
			\texttt{Ctrl+Z} & Suspend current process \\
			\texttt{Ctrl+D} & Exit shell/logout \\
			\texttt{Ctrl+L} & Clear screen \\
			\texttt{Ctrl+A} & Go to beginning of line \\
			\texttt{Ctrl+E} & Go to end of line \\
			\texttt{Ctrl+R} & Search command history \\
			\texttt{Up/Down arrows} & Navigate command history \\
			\texttt{Tab} & Auto-complete filenames/commands \\
			\bottomrule
		\end{tabular}
		\caption{Keyboard shortcuts for terminal efficiency}
	\end{table}
	
	\subsection{Wildcards and Pattern Matching}
	
	\begin{lstlisting}[style=terminal, caption=Using wildcards]
		# List all .txt files
		ls *.txt
		
		# Remove all backup files
		rm *.bak
		
		# List files starting with 'data' followed by any characters
		ls data*
		
		# List files with single-character wildcard
		ls data?.csv  # matches data1.csv, data2.csv, but not data10.csv
	\end{lstlisting}
	
	\subsection{Input/Output Redirection}
	
	\begin{table}[h]
		\centering
		\begin{tabular}{lll}
			\toprule
			\textbf{Operator} & \textbf{Meaning} & \textbf{Example} \\
			\midrule
			\terminalcmd{>} & Redirect output to file (overwrite) & \terminalcmd{ls > filelist.txt} \\
			\terminalcmd{>>} & Redirect output to file (append) & \terminalcmd{echo "new" >> file.txt} \\
			\terminalcmd{<} & Take input from file & \terminalcmd{sort < unsorted.txt} \\
			\terminalcmd{|} & Pipe output to another command & \terminalcmd{ls -la | grep "txt"} \\
			\bottomrule
		\end{tabular}
		\caption{Redirection and pipes}
	\end{table}
	
	\begin{lstlisting}[style=terminal, caption=Pipe examples]
		# Count files in directory
		ls -1 | wc -l
		
		# Find specific files
		ls -la | grep "cfd"
		
		# Sort and display unique
		cat data.txt | sort | uniq
	\end{lstlisting}
	
	\subsection{Process Management}
	
	\begin{lstlisting}[style=terminal, caption=Managing running processes]
		# List running processes
		ps aux
		
		# Interactive process viewer
		htop
		
		# Run command in background
		./longsimulation.sh &
		
		# Bring background job to foreground
		fg
		
		# Kill a process
		kill -9 PID
	\end{lstlisting}
	
	% ========== SECTION 3: BASH SCRIPTING ==========
	\section{Introduction to Bash Scripting}
	
	\subsection{What is a Shell Script?}
	A shell script is a text file containing a series of commands that can be executed as a program. This automates repetitive tasks.
	
	\subsection{Your First Script}
	
	\begin{enumerate}
		\item Create a new file:
		\begin{lstlisting}[style=terminal]
			vim firstscript.sh
		\end{lstlisting}
		
		\item Add the following content:
		\begin{lstlisting}[style=terminal, caption=hello.sh]
			#!/bin/bash
			# This is a comment - my first script
			
			echo "Hello, CFD world!"
			echo "Current directory: $(pwd)"
			echo "Files here:"
			ls -la
		\end{lstlisting}
		
		\item Make it executable:
		\begin{lstlisting}[style=terminal]
			chmod +x firstscript.sh
		\end{lstlisting}
		
		\item Run it:
		\begin{lstlisting}[style=terminal]
			./firstscript.sh
		\end{lstlisting}
	\end{enumerate}
	
	\important{The line \terminalcmd{\#!/bin/bash} (shebang) tells the system which interpreter to use. Without it, the script won't execute properly.}
	
	\subsection{Variables in Bash}
	
	\begin{lstlisting}[style=terminal, caption=variables.sh]
		#!/bin/bash
		
		# Defining variables (no spaces around =)
		name="Student"
		course="CFD"
		simulation_time=3600
		
		# Using variables (with $)
		echo "Hello, $name!"
		echo "Welcome to $course course."
		
		# Command substitution - store command output
		files=$(ls -la)
		echo "Files in current directory: $files"
		
		# Arithmetic
		a=10
		b=20
		sum=$((a + b))
		echo "Sum: $sum"
	\end{lstlisting}
	
	\subsection{Special Variables}
	
	\begin{table}[H]
		\centering
		\begin{tabular}{ll}
			\toprule
			\textbf{Variable} & \textbf{Meaning} \\
			\midrule
			\texttt{\$0} & Script name \\
			\texttt{\$1}, \texttt{\$2}, ... & Positional parameters (arguments) \\
			\texttt{\$\#} & Number of arguments \\
			\texttt{\$*} & All arguments as single string \\
			\texttt{\$@} & All arguments as separate strings \\
			\texttt{\$?} & Exit status of last command \\
			\texttt{\$\$} & Process ID of current script \\
			\bottomrule
		\end{tabular}
		\caption{Special shell variables}
	\end{table}
	
	\begin{lstlisting}[style=terminal, caption=arguments.sh]
		#!/bin/bash
		echo "Script name: $0"
		echo "First argument: $1"
		echo "Second argument: $2"
		echo "Number of arguments: $#"
		echo "All arguments: $@"
		
		if [ $# -eq 0 ]; then
		echo "Error: No arguments provided!"
		exit 1
		fi
	\end{lstlisting}
	
	\subsection{Conditional Statements}

	\begin{table}[H]
	\centering
	\begin{tabular}{ll}
		\toprule
		\textbf{Operator} & \textbf{Meaning} \\
		\midrule
		\multicolumn{2}{l}{\textbf{File tests}} \\
		\terminalcmd{-e file} & File exists \\
		\terminalcmd{-f file} & File exists and is regular file \\
		\terminalcmd{-d file} & File exists and is directory \\
		\multicolumn{2}{l}{\textbf{Numeric comparisons}} \\
		\terminalcmd{-eq} & Equal to \\
		\terminalcmd{-ne} & Not equal to \\
		\terminalcmd{-lt} & Less than \\
		\terminalcmd{-gt} & Greater than \\
		\multicolumn{2}{l}{\textbf{String comparisons}} \\
		\terminalcmd{=} & Strings equal \\
		\terminalcmd{!=} & Strings not equal \\
		\terminalcmd{-z} & String is empty \\
		\bottomrule
	\end{tabular}
	\caption{Common test operators}
\end{table}
	
	\begin{lstlisting}[style=terminal, caption=conditions.sh]
		#!/bin/bash
		
		# File tests
		if [ -e "mesh.geo" ]; then
		echo "mesh.geo exists"
		else
		echo "mesh.geo not found"
		fi
		
		# String comparison
		name="OpenFOAM"
		if [ "$name" = "OpenFOAM" ]; then
		echo "Correct solver"
		fi
		
		# Numeric comparison
		value=10
		if [ $value -gt 5 ]; then
		echo "Value is greater than 5"
		fi
	\end{lstlisting}
	
	
	\subsection{Loops}
	
	\begin{lstlisting}[style=terminal, caption=loops.sh]
		#!/bin/bash
		
		# For loop over explicit values
		echo "=== For loop with values ==="
		for fruit in apple banana orange; do
		echo "I like $fruit"
		done
		
		# For loop with range
		echo "=== For loop with range ==="
		for i in {1..5}; do
		echo "Iteration $i"
		done
		
		# For loop over files
		echo "=== For loop over files ==="
		for file in *.txt; do
		echo "Processing $file"
		wc -l "$file"
		done
		
		# While loop
		echo "=== While loop ==="
		counter=1
		while [ $counter -le 5 ]; do
		echo "Counter: $counter"
		counter=$((counter + 1))
		done
	\end{lstlisting}
	
	\subsection{Reading User Input}
	
	\begin{lstlisting}[style=terminal, caption=input.sh]
		#!/bin/bash
		
		# Basic input
		echo -n "Enter your name: "
		read name
		echo "Hello, $name!"
		
		# Prompt with message
		read -p "Enter Reynolds number: " Re
		echo "Reynolds number: $Re"
		
		# Hidden input (for passwords)
		read -s -p "Enter password: " password
		echo "Password accepted."
		
		# Read multiple values
		read -p "Enter x y z coordinates: " x y z
		echo "Coordinates: ($x, $y, $z)"
	\end{lstlisting}
	
	% ========== SECTION 4: GIT BASICS ==========
	\section{Version Control with Git}
	
	\subsection{What is Git?}
	Git is a distributed version control system created by Linus Torvalds in 2005 for Linux kernel development. It tracks changes to files, enables collaboration, and maintains complete history of your projects.
	
	\textbf{Why Git for CFD?}
	\begin{itemize}
		\item Track changes to simulation setups
		\item Collaborate with team members
		\item Revert to working versions when experiments fail
		\item Share cases with reproducibility
	\end{itemize}
	
	\subsection{Installing and Configuring Git}
	
	\begin{lstlisting}[style=terminal, caption=Install Git]
		sudo apt install git
		git --version  # Verify installation
	\end{lstlisting}
	
	\subsection{First-Time Setup}
	
	\begin{lstlisting}[style=terminal, caption=Configure Git]
		# Set your identity (required for commits)
		git config --global user.name "Your Name"
		git config --global user.email "your.email@example.com"
		
		# Set default editor
		git config --global core.editor "vim"
		
		# View all settings
		git config --list
	\end{lstlisting}
	
	\subsection{Creating Your First Repository}
	
	\begin{lstlisting}[style=terminal, caption=Initialize a repository]
		# Create project directory
		mkdir pitzDaily_case
		cd pitzDaily_case
		
		# Initialize Git repository
		git init
		
		# Check status (should show empty repo)
		git status
	\end{lstlisting}
	
	After \terminalcmd{git init}, you'll have a hidden \texttt{.git} directory containing all version control information.
	
	\subsection{Basic Git Workflow}
	
	The typical Git workflow has three stages:
	
	\begin{enumerate}
		\item \textbf{Working Directory}: Where you edit files
		\item \textbf{Staging Area} (index): Where you prepare changes
		\item \textbf{Repository}: Where commits are permanently stored
	\end{enumerate}
	
	\begin{figure}[h]
		\centering
		\begin{verbatim}
			Working Directory    git add    Staging Area    git commit    Repository
			(files)             -------->    (index)         -------->    (commits)
			^                                                                |
			|                                                                 |
			+----------------------- git checkout <--------------------------+
		\end{verbatim}
		\caption{Git workflow stages}
	\end{figure}
	
	\subsection{Tracking Changes}
	
	\begin{lstlisting}[style=terminal, caption=First commit]
		# Create a file
		echo "# pitzDaily Simulation" > README.md
		echo "U uniform (10 0 0);" > 0/U
		
		# Check status
		git status
		
		# Stage files
		git add README.md
		git add 0/U
		
		# Or stage all changes
		git add .
		
		# Check status again (files now staged)
		git status
		
		# Commit with message
		git commit -m "Initial case setup: README and velocity BC"
		
		# View commit history
		git log --oneline
	\end{lstlisting}
	
	\subsection{Checking Differences}
	
	\begin{lstlisting}[style=terminal, caption=Viewing changes]
		# See unstaged changes
		git diff
		
		# See staged changes
		git diff --staged
		
		# Compare with previous commit
		git diff HEAD^
		
		# Show specific commit
		git show COMMIT_HASH
	\end{lstlisting}
	
	\subsection{Viewing History}
	
	\begin{lstlisting}[style=terminal, caption=git log examples]
		# Simple log
		git log --oneline
		
		# Detailed log with graph
		git log --graph --oneline --decorate --all
		
		# Show last 3 commits
		git log -3
		
		# Search commits by message
		git log --grep="mesh refinement"
	\end{lstlisting}
	
	The output shows commit hash, author, date, and message:
	\begin{lstlisting}[style=output]
		abc1234 (HEAD -> main) Update boundary conditions
		def5678 Add mesh refinement study
		ghi9012 Initial commit
	\end{lstlisting}
	
	\subsection{Ignoring Files}
	
	Create \filename{.gitignore} to exclude files from version control:
	
	\begin{lstlisting}[style=terminal, caption=.gitignore]
		# Ignore OpenFOAM processor directories
		processor*/
		
		# Ignore log files
		*.log
		log.*
		
		# Ignore backup files
		*~
		*.bak
		
		# Ignore large result files
		*.vtk
		*.foam
		
		# Ignore IDE files
		.vscode/
		.idea/
	\end{lstlisting}
	
	\subsection{Undoing Changes}
	
	\begin{lstlisting}[style=terminal, caption=Fixing mistakes]
		# Unstage a file (keep changes)
		git reset HEAD filename
		
		# Discard changes in working directory
		git checkout -- filename
		
		# Amend last commit (fix message or add forgotten files)
		git commit --amend -m "Better commit message"
		
		# Revert to previous commit (creates new commit)
		git revert HEAD
	\end{lstlisting}
	
	\warning{Be careful with \terminalcmd{git reset} and \terminalcmd{git checkout} as they can permanently discard changes!}
	
	
%	Remote repository is an advanced topic that will be explained later...
%	
%	\subsection{Working with Remotes (GitHub)}
%	
%	\begin{enumerate}
%		\item \textbf{Create GitHub account}: \url{https://github.com}
%		\item \textbf{Create new repository} on GitHub (don't initialize with README)
%		\item \textbf{Connect local repository}:
%	\end{enumerate}
%	
%	\begin{lstlisting}[style=terminal, caption=Adding remote]
%		# Add remote repository
%		git remote add origin git@github.com:<username>/pitzDaily_case.git
%		
%		# Rename main branch
%		git branch -M main
%		
%		# Verify remote
%		git remote -v
%		
%		# Push to GitHub
%		git push -u origin main
%	\end{lstlisting}
	
	\subsection{Cloning and Pulling}
	
	\begin{lstlisting}[style=terminal, caption=Working with existing repositories]
		# Clone a repository
		git clone https://github.com/username/pitzDaily_case.git
		
		# Get latest changes
		git pull
		
		# Fetch without merging
		git fetch
	\end{lstlisting}
	
	\subsection{Branching (Advanced)}
	
	Branches allow parallel development:
	
	\begin{lstlisting}[style=terminal, caption=Branch basics]
		# List branches
		git branch
		
		# Create new branch
		git branch mesh-refinement
		
		# Switch to branch
		git checkout mesh-refinement
		
		# Create and switch in one command
		git checkout -b turbulence-study
		
		# Merge branch into main
		git checkout main
		git merge mesh-refinement
		
		# Delete branch
		git branch -d mesh-refinement
	\end{lstlisting}
	
	\subsection{Removing the local repository}
	
	The local repository can be easily removed by just deleting the \filename{.git} folder.
	
	\begin{lstlisting}[style=terminal, caption=Branch basics]
		# Removing git repository
		rm -rf .git
	\end{lstlisting}
	
	\tip{The \terminalcmd{-f} option is to \texttt{force} the removing without asking 
		for confirmation}
	\warning{Be cautious with this command. It can not be undone!}
	
	It is also possible to work with remote repositories, \href{https://www.github.com}{GitHub} for instance, but
	it is an advanced topic that is out of scope of the present Tutorial
	
	
	\section{Simple computations with python}
	
	\subsection{What is python?}
	
	Python is a high-level computation language. It is simular to Matlab, but it 
	is open-source and free. You can make from simple mathematical operations to 
	very complex program, by using the huge repositoty of modules and packages 
	available.
	
	\textbf{Why python for CFD?}
	\begin{itemize}
		\item Data analysis of results
		\item Automation of tasks
	\end{itemize}
	
	\subsection{Installation of python and an IDE (Integrated Development Enviroment)}
	
	By default, python version 3 is installed in Ubuntu 24.04
	
	\begin{lstlisting}[style=terminal, caption=python basics]
		# python version
		python3 --version
		
		# basic interface for python
		python3
	\end{lstlisting}
	
	However, it is preferable to make a personal environment and install an IDE. There are
	several options for an IDE for python: 
	\begin{enumerate}
		\item \href{https://jupyter.org/}{Jupyter Lab}
		\item \href{https://www.spyder-ide.org/}{Spyder}
		\item \href{https://www.jetbrains.com/pycharm/}{PyCharm}
		\item \href{https://colab.research.google.com/}{Google Colab}
		\item ...
	\end{enumerate}
	
	All these IDE's have pros and cons. In this 
	tutorial we install and configure the web browser version of Jupyter Lab.
	
	First, install all the needed packages
	
	\begin{lstlisting}[style=terminal, caption=Installing python packages and creating the environment]
		# packages installation
		sudo apt install  python3-pip python3-venv python3-full
		
		# Make the local virtual enviroment
		python3 -m venv my_env
		
		# Activate the enviroment
		source ./my_venv/bin/activate 
	\end{lstlisting}
	
	After the activation of the virtual environment, you will get a \texttt{(my\_env)} 
	at the beginning of the prompt in the terminal, meaning that you are working in 
	a local virtual environment (not in the global python system) and all the python
	modeules installed will be in this local venv.
	
	\tip{You can create as much virtual enviroments as you want. Each one of these 
	\texttt{venv}'s can have different modules and/or different versions.}
	
	To exit form the local virtual enviroment, just deactivate it
	
	\begin{lstlisting}[style=terminal, caption=Deactivation of the virtual environment]
		# venv deactivation
		deactivate
	\end{lstlisting}
	
	Install the usual modules needed for Engineering computations in python. Be sure to have the virtual enviroment activated.
	
	\begin{lstlisting}[style=terminal, caption=Installing python modules and running jupyter lab]
		# modules installation
		pip install jupyter numpy scipy matplotlib pandas
		
		# Run jupyter
		jupyter lab
	\end{lstlisting}
	
	Jupyter Lab runs in the web broser with a local server. 
	
	% ========== SECTION 5: PRACTICE EXERCISES ==========
	\section{Practice Exercises}
	
	\subsection{Exercise 1: Filesystem Navigation}
	\begin{enumerate}
		\item Create directory structure: \terminalcmd{cfd-course/tutorials/pitzDaily}
		\item Navigate to this directory
		\item Create files: \texttt{README.md}, \texttt{0/U}, \texttt{constant/transportProperties}
		\item List all files recursively
		\item Count number of files in the project
	\end{enumerate}
	
	\subsection{Exercise 2: Bash Scripting}
	Create a script \texttt{setup\_case.sh} that:
	\begin{enumerate}
		\item Checks if directory exists, creates if not
		\item Prompts user for Reynolds number
		\item Creates basic OpenFOAM file structure
		\item Prints summary of what was created
	\end{enumerate}
	
	Solution template:
	\begin{lstlisting}[style=terminal]
		#!/bin/bash
		# CFD case setup script
		
		case_name=$1
		if [ -z "$case_name" ]; then
		read -p "Enter case name: " case_name
		fi
		
		# Your code here...
	\end{lstlisting}
	
	\subsection{Exercise 3: Git Practice}
	\begin{enumerate}
		\item Initialize Git repository in your case directory
		\item Create and commit initial files
		\item Make changes, view differences
		\item View commit history
		\item Create \texttt{.gitignore} for CFD files
%		\item Push to GitHub (create account if needed)
	\end{enumerate}
	
	\subsection{Exercise 4: Combine Skills}
	Create a script that:
	\begin{enumerate}
		\item Creates a new Git repository
		\item Sets up standard CFD case structure
		\item Makes initial commit
		\item Displays repository status and log
	\end{enumerate}
	
	% ========== SECTION 6: TROUBLESHOOTING ==========
	\section{Common Issues and Solutions}
	
	\subsection{Permission Denied}
	\begin{lstlisting}[style=terminal]
		# Make script executable
		chmod +x script.sh
		
		# Check permissions
		ls -la script.sh
	\end{lstlisting}
	
	\subsection{Command Not Found}
	\begin{lstlisting}[style=terminal]
		# Check if installed
		which <command>
		
		# Install if missing
		sudo apt install <package>
	\end{lstlisting}
	
%	\subsection{Git Push Rejected}
%	\begin{lstlisting}[style=terminal]
%		# Pull changes first
%		git pull origin main
%		
%		# Resolve conflicts if any
%		# Then push again
%		git push origin main
%	\end{lstlisting}
	
	% ========== SECTION 7: RESOURCES ==========
	\section{Additional Resources}
	
	\subsection{Online Tutorials}
	\begin{itemize}
		\item \href{https://ubuntu.com/tutorials}{Ubuntu Official Tutorials}
		\item \href{https://southampton-rsg-training.github.io/shell-novice/}{The Bash Shell} - Southampton course
		\item \href{https://git-scm.com/doc}{Official Git Documentation}
	\end{itemize}
	
	\subsection{Cheat Sheets}
	\begin{itemize}
		\item \href{https://ubunlog.com/wp-content/uploads/2025/11/Ubuntu-Server-CLI-cheat-sheet-2024-v6.pdf}{Ubuntu command line interface (CLI)}
		\item \href{https://help.ubuntu.com/stable/ubuntu-help/shell-keyboard-shortcuts.html.en}{Ubuntu keyboard shortcuts}
		\item \href{https://linuxize.com/cheatsheet/bash/}{Bash commands reference}
		\item \href{https://git-scm.com/cheat-sheet}{Git quick reference}
	\end{itemize}
	
	\subsection{Books}
	\begin{itemize}
		\item \href{https://landing.packtpub.com/the-ultimate-ubuntu-handbook/}
		{"The Ultimate Ubuntu Handbook"} by Ken VanDine
		\item \href{https://link.springer.com/book/10.1007/978-1-4842-8972-3}{"Beginning Ubuntu for Windows and Mac Users"} by Nathan Haines
		\item \href{https://git-scm.com/book/en/v2}{"Pro Git"} by Scott Chacon (free online)
	\end{itemize}
	
	\subsection{Community}
	\begin{itemize}
		\item \href{https://askubuntu.com}{Ask Ubuntu}
		\item \href{https://stackoverflow.com}{Stack Overflow} (tag: bash, git)
		\item \href{https://discourse.ubuntu.com}{Ubuntu Discourse}
	\end{itemize}
	
%	% ========== CONCLUSION ==========
%	\section{What's Next?}
%	
%	You're now ready for:
%	\begin{itemize}
%		\item \textbf{Tutorial \#1}: OpenFOAM terminal workflow with pitzDaily
%		\item \textbf{Tutorial \#2}: FreeCAD and CfdOF GUI workflow
%		\item Advanced CFD topics
%	\end{itemize}
	
	\important{Keep practicing! The command line becomes intuitive with regular use.}
	
	% ========== APPENDIX ==========
	\appendix
	\section{Quick Reference Cards}
	
	\subsection{Ubuntu Shortcuts}
	\begin{tabular}{ll}
		\texttt{Ctrl + Alt + D} & Show desktop \\
		\texttt{Alt + Tab} & Switch applications \\
		\texttt{Ctrl + Alt + T} & Open terminal \\
		\texttt{PrtScn} & Take screenshot \\
	\end{tabular}
	
	\subsection{Bash Commands Summary}
	\begin{tabular}{ll}
		\terminalcmd{ls -la} & List all files with details \\
		\terminalcmd{cp -r} & Copy recursively \\
		\terminalcmd{grep <pattern> <file>} & Search in files \\
		\terminalcmd{history} & Show command history \\
		\terminalcmd{man <command>} & Show manual \\
		\terminalcmd{which <program>} & Locate program \\
	\end{tabular}
	
	\subsection{Git Commands Summary}
	\begin{tabular}{ll}
		\terminalcmd{git init} & Initialize repository \\
		\terminalcmd{git add} & Stage changes \\
		\terminalcmd{git commit -m "msg"} & Commit staged changes \\
		\terminalcmd{git status} & Check status \\
		\terminalcmd{git log} & View history \\
		\terminalcmd{git diff} & Show changes \\
%		\terminalcmd{git push} & Upload to remote \\
%		\terminalcmd{git pull} & Download from remote \\
	\end{tabular}
	
\vspace{1cm}
\noindent
\textbf{Acknowledgments:} The author acknowledges the assistance of 
\href{https://deepseek.com/en}{DeepSeek}, an AI assistant created by DeepSeek Company, 
for providing valuable support in structuring this document, formatting LaTeX code, 
and developing the tutorial content.

\textbf{Disclaimer:} All content has been reviewed, modified, and 
validated by the author. The author takes full responsibility for 
the accuracy, completeness, and educational quality of this material. 


\end{document}